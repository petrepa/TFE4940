\chapter{Qualitative Feedback}\label{cha:appendix-qualitative}


\begin{longtable}{p{0.5cm}p{11cm}}
\caption{Qualitative Feedback}
\label{tab:qualitative_feedback}\\
    \hline
        1 & The examples are good \\
        \hline
        2 & Likte dansen \\
        \hline
        3 & Undersøkelse med høg kvalitet.\\
        \hline
        4 & Vet ikke helt hva jeg svarte på her men du tar deg godt ut på video :) milla \\
        \hline
        5 & Imponerende teknologi. Ser at små detaljer er vanskeligere å trekke ut enn større. Eksperimentet med boken er også interessant. Siden denne er større en fingrene, burde det vært mindre klipping på den, men det kan ha med lys, farge og refleksjon i overflaten på boken. Uansett en veldig imponerende teknologi, da jeg regner med dette er gjort uten green-screen \\
        \hline
        6 & Generelt meir irriterande på dei videoane der bakgrunnen tidvis flimrar inn i bildet - veldig "visuelt" forstyrrande. Der det manglar ein finger eller to oppfattast som mindre irriterande, så lenge feilen "vedvarer" (ikkje blinkar/flimrar). Hjerna veit jo på ein måte at fingeren er der? PS. Masse lykke til med vidare arbeid med masteren! Hang in there :-)\\
        \hline
        7 & Det var mykje lik kvalitet på videoane. Eg var stort sett ikkje fornøgd med nåken av dei. \\
        \hline
        8 & Veldig bra! Mest minus til det som forsvinner, feks hvis man skal vise frem forsiden av en bok og den blir "filtrert bort". Ikke like farlig/annoying med litt artefacts som henger igjen etter bevegelse.\\
        \hline
        9 & usikker på om eg skjønte spørsmåla, men trur det :) \\
        \hline
        10 & Morsom undersøkelse, bra jobbet! Den eneste tilbakemeldingen jeg har er at spørsmålene kanskje var litt for tekniske og det derfor var litt vanskelig å være sikker på at man skjønte hva du spør om. En liten introtekst til hva det handler om kunne vært fint. Da kunne du også definert noen begreper så alle vet hva det blir stilt spørsmål om. Evt skrive spørsmålene litt mer sånn som man ville snakket til en 5-åring. Men jeg likte undersøkelsen godt, det var gøy! Lykke til videre :) \\
        \hline
        11 & Ser ut som silhuettene skildres greit ut på kroppen, men noe flimring rundt fingrer og armer. \\
        \hline
        12 & Sakna ein piruett eller to, elles flott jobba!\\
        \hline
        13 & Kult prosjekt, ønsker mer variasjon i dansemoves til neste gang.\\
        \hline
        14 & Dette var et arti eksperiment med godt gjennomført spørreskjema og gode svaralternativer. Lurte kun på om fargene på klærne burde vært den samme? \\
        \hline
        15 & Blinking er mest irriterende \\
        \hline
        16 & Svært interessant :D \\
        \hline
        17 & Nice presentation. Impressed our the video quality. \\
        \hline
        18 & Seems that in some videos the silhouette extractor works very well, but on what seems to be identical tests the extractor also struggles a lot, even when the object wears identical clothes. \\
        \hline
        19 & V beautiful videos. This could be an art exhibition<33 \\
        \hline
        20 & Interesting work! Looking forward to read the final thesis \\
        \hline
\end{longtable}